
%
\documentclass[12pt,oneside,letterpaper]{report}
\usepackage[spanish]{babel}
\usepackage[ansinew]{inputenc}
\usepackage[dvips]{graphicx}
\usepackage[right=2cm,left=3cm,top=2cm,bottom=2cm,headsep=0cm,footskip=0.5cm]{geometry}

%-------------------------------------------------------------------------
%definici�n de comandos

\newcommand{\dotrule}[1]{\parbox[t]{#1}{\dotfill}}

%para c�digo fuente
\newenvironment{codigoenv}
{\fontsize{10pt}{12pt} \linespread{1}} { \normalsize}

%para las imagenes
\newenvironment{figuraenv}
{\begin{figure}[htb]\begin{center}} {\end{center}\end{figure}}


\newcommand{\df}[2]{\textit{#1 (#2)}} %definicion de termino y sigla
\newcommand{\cls}[1]{\mbox{\textit{#1}}} %nombre de clase o paquete o c�digo
\newcommand{\trm}[1]{\textit{#1}} %termino tecnico o en ingles
\newcommand{\cod}[1]{\texttt{\footnotesize #1}} %codigo fuente

%-------------------------------------------------------------------------


\linespread{1}
\setlength{\parskip}{1\baselineskip}
\parindent 1cm
\sloppy


%-------------------------------------------------------------------------

\begin{document}

%-------------------------------------------------------------------------
\thispagestyle{empty}
\begin{center}

UNIVERSIDAD DE CHILE\\
FACULTAD DE CIENCIAS F�SICAS Y MATEM�TICAS\\
DEPARTAMENTO DE CIENCIAS DE LA COMPUTACI�N\\

\vspace{4cm}

CC4102 - Dise�o y Analisis de Algoritmos

\vspace{3cm}

Tarea 1: Diccionarios1



\vspace{9cm}
\begin{flushright}
\begin{tabular}{l}
Cristian Carre�o Medina\\
Diego Ch�vez Escobar\\
Prof. Jeremy Barbay; Aux. Mauricio Quezada
\end{tabular}
\end{flushright}
\vfill


\end{center}

%-------------------------------------------------------------------------

%-------------------------------------------------------------------------

%-------------------------------------------------------------------------
\newpage \pagenumbering{roman}
\tableofcontents

%-------------------------------------------------------------------------


%-------------------------------------------------------------------------
\newpage
\pagenumbering{arabic}
\chapter{Presentaci�n}

\section{Introducci�n}

En el presente informe se detallaran los resultados en terminos de eficiencia de cuatro estructuras de datos, se pondran a prueba las funciones de insercion, busqueda y eliminacion de datos, en las siguientes estructuras:
\begin{enumerate}
\vspace{1cm}
\item
Arbol AVL
\item
Arbol Rojo-Negro
\item
Arbol 2-3
\item
Arbol B
\end{enumerate}


%-------------------------------------------------------------------------
\newpage
\chapter{Experimentacion}


%-------------------------------------------------------------------------
\section{Marco Teorico}

\subsection{Estructuras}

\subsubsection{Arbol AVL}

Estos Arboles est�n siempre equilibrados de tal modo que para todos los nodos, la altura de la rama izquierda no difiere en m�s de una unidad de la altura de la rama derecha o viceversa. 
Gracias a esta forma de equilibrio (o balanceo), la complejidad de una b�squeda en uno de estos �rboles se mantiene siempre en orden de complejidad O(log n). 
El factor de equilibrio puede ser almacenado directamente en cada nodo o ser computado a partir de las alturas de los sub�rboles.


\subsubsection{Arbol Rojo-Negro}

Un �rbol rojo-negro es un �rbol binario de b�squeda en el que cada nodo tiene un atributo de color cuyo valor es o bien rojo o bien negro, las hojas poseen valor nulos, los hijos de todo nodo rojo es negro, y el valor de nodos negros en el camino es constante independiente del camino, con esto se logra un arbol aproximadamente equilibrado.

\subsubsection{Arbol 2-3}

Un �rbol 2-3 permite que un nodo tenga dos o tres hijos ademas que todas las hojas han de estar al mismo nivel. Esta caracter�stica le permite conservar el balanceo tras insertar o borrar elementos, por lo que el algoritmo de b�squeda es casi tan r�pido como en un �rbol de b�squeda de altura m�nima. Por otro lado, es mucho m�s f�cil de mantenerlo.

\subsubsection{Arbol B}

Los �rboles-B poseen un numero variable de nodos hijos con un maximo predifinido. Cuando se inserta o se elimina un dato de la estructura, la cantidad de nodos hijo var�a dentro de un nodo. Para que siga manteni�ndose el n�mero de nodos dentro del rango predefinido, los nodos internos se juntan o se parten. Dado que se permite un rango variable de nodos hijo, los �rboles-B no necesitan rebalancearse tan frecuentemente como los �rboles binarios de b�squeda auto-balanceables, pero por otro lado pueden desperdiciar memoria, porque los nodos no permanecen totalmente ocupados. Los l�mites superior e inferior en el n�mero de nodos hijo son definidos para cada implementaci�n en particular.

\section{Dise�o Experimental}

Lorem ipsum dolor sit amet, consectetur adipiscing elit. Nulla ullamcorper mauris vel nulla egestas sit amet lobortis mi fermentum. Donec elit tellus, feugiat non facilisis non, faucibus sit amet dolor. Fusce feugiat imperdiet mi, ut iaculis eros lacinia vel. Phasellus vel erat eget metus iaculis adipiscing et eu nulla. Cras nec ligula et magna tristique tempus quis sed magna. Nullam non orci at est dignissim auctor at at dui. Vestibulum molestie ultricies libero, eget sollicitudin metus venenatis eu.

In quis tempor risus. Mauris dapibus porttitor nisl, nec lobortis velit fermentum in. Sed fermentum suscipit orci, at porta lorem fringilla in. Pellentesque et arcu et risus vehicula imperdiet. Sed in lorem ac turpis aliquet laoreet a et ante. Nunc ac libero quis diam semper accumsan. Proin semper lectus sed neque volutpat eget scelerisque dolor placerat. 

%-------------------------------------------------------------------------

\newpage
\chapter{Resultados y Analisis}


%-------------------------------------------------------------------------
\section{Resultados}

\subsection{Arbol AVL}
\subsubsection{Prueba 1}
Lorem ipsum dolor sit amet, consectetur adipiscing elit. Nulla ullamcorper mauris vel nulla egestas sit amet lobortis mi fermentum.
\subsubsection{Prueba 2}
Lorem ipsum dolor sit amet, consectetur adipiscing elit. Nulla ullamcorper mauris vel nulla egestas sit amet lobortis mi fermentum.
\subsubsection{Prueba 3}
Lorem ipsum dolor sit amet, consectetur adipiscing elit. Nulla ullamcorper mauris vel nulla egestas sit amet lobortis mi fermentum.

\subsection{Arbol Rojo-Negro}
\subsubsection{Prueba 1}
Lorem ipsum dolor sit amet, consectetur adipiscing elit. Nulla ullamcorper mauris vel nulla egestas sit amet lobortis mi fermentum.
\subsubsection{Prueba 2}
Lorem ipsum dolor sit amet, consectetur adipiscing elit. Nulla ullamcorper mauris vel nulla egestas sit amet lobortis mi fermentum.
\subsubsection{Prueba 3}
Lorem ipsum dolor sit amet, consectetur adipiscing elit. Nulla ullamcorper mauris vel nulla egestas sit amet lobortis mi fermentum.

\subsection{Arbol 2-3}
\subsubsection{Prueba 1}
Lorem ipsum dolor sit amet, consectetur adipiscing elit. Nulla ullamcorper mauris vel nulla egestas sit amet lobortis mi fermentum.
\subsubsection{Prueba 2}
Lorem ipsum dolor sit amet, consectetur adipiscing elit. Nulla ullamcorper mauris vel nulla egestas sit amet lobortis mi fermentum.
\subsubsection{Prueba 3}
Lorem ipsum dolor sit amet, consectetur adipiscing elit. Nulla ullamcorper mauris vel nulla egestas sit amet lobortis mi fermentum.

\subsection{Arbol B}
\subsubsection{Prueba 1}
Lorem ipsum dolor sit amet, consectetur adipiscing elit. Nulla ullamcorper mauris vel nulla egestas sit amet lobortis mi fermentum.
\subsubsection{Prueba 2}
Lorem ipsum dolor sit amet, consectetur adipiscing elit. Nulla ullamcorper mauris vel nulla egestas sit amet lobortis mi fermentum.
\subsubsection{Prueba 3}
Lorem ipsum dolor sit amet, consectetur adipiscing elit. Nulla ullamcorper mauris vel nulla egestas sit amet lobortis mi fermentum.

\section{Analisis de  Resultados}

Lorem ipsum dolor sit amet, consectetur adipiscing elit. Nulla ullamcorper mauris vel nulla egestas sit amet lobortis mi fermentum. Donec elit tellus, feugiat non facilisis non, faucibus sit amet dolor. Fusce feugiat imperdiet mi, ut iaculis eros lacinia vel. Phasellus vel erat eget metus iaculis adipiscing et eu nulla. Cras nec ligula et magna tristique tempus quis sed magna. Nullam non orci at est dignissim auctor at at dui. Vestibulum molestie ultricies libero, eget sollicitudin metus venenatis eu.



%-------------------------------------------------------------------------
\newpage
\chapter{Conclusiones}
\section{Conclusiones}

Lorem ipsum dolor sit amet, consectetur adipiscing elit. Nulla ullamcorper mauris vel nulla egestas sit amet lobortis mi fermentum. Donec elit tellus, feugiat non facilisis non, faucibus sit amet dolor. Fusce feugiat imperdiet mi, ut iaculis eros lacinia vel. Phasellus vel erat eget metus iaculis adipiscing et eu nulla. Cras nec ligula et magna tristique tempus quis sed magna. Nullam non orci at est dignissim auctor at at dui. Vestibulum molestie ultricies libero, eget sollicitudin metus venenatis eu.

In quis tempor risus. Mauris dapibus porttitor nisl, nec lobortis velit fermentum in. Sed fermentum suscipit orci, at porta lorem fringilla in. Pellentesque et arcu et risus vehicula imperdiet. Sed in lorem ac turpis aliquet laoreet a et ante. Nunc ac libero quis diam semper accumsan. Proin semper lectus sed neque volutpat eget scelerisque dolor placerat. 

%-------------------------------------------------------------------------

%-------------------------------------------------------------------------
\end{document}
